% Very simple template for lab reports. Most common packages are already included.
\documentclass[a4paper, 11pt]{article}
\usepackage[utf8]{inputenc} % Change according your file encoding
\usepackage{graphicx}
\usepackage{url}

%opening
\title{Report 2: Routy}
\author{Anton Blomberg}
\date{\today{}}

\begin{document}

\maketitle

\section{Introduction}

\textit{Implementing a software network router using the Link-state protocol}

A router by itself is nothing but a piece of code standing at a halt. On the other hand, a network of interconnected
routers is a good example of a distributed system. As long as each router follows the interface of the Link-state
protocol each router can run it's own implementation. The code on each router runs independently and a network is formed
by exchanging messages.
A network with independently running processes make for a maintainable system since it will adapt itself to it's
environment instead of relying on a predefined set of required resources or services. A router that doesn't expect or
assume a specific state of other routers in a network can handle situations like network reconfiguration, link/router
failure or other external factors.


\section{Main problems and solutions}

\textit{The main problem of creating a truly distributed system of independently running processes is verifying its
correctness in the presence of concurrency and race conditions.}

An error in a single procedure can be carried over the network and corrupting the state of another process which makes
debugging and reasoning about the system hard. The link-state protocol offers a complete picture of the network which
 makes this task easier by letting an observer verify the full state of each routers view of the network.
Another problem present in networked systems is the possibility of failure, and the problem of detecting it. Through
 Erlang processes can be monitored, but one can still not be 100\% sure about the state of a remote process not
 answering.

\pagebreak
To verify the correctness of the system I had to create a test case simple enough to analyze, but complex enough to
spot any errors. A simplified and not geographically correct map of Sweden was created, represented by a undirected
graph like this:

\begin{verbatim}
Göteborg ------- Örebro
    \                \------ Uppsala
     \                          \
      \---------- Lund           \
                    \---------Stockholm
\end{verbatim}

This graph was created by creating interfaces in both directions between all connected cities like so:
\begin{verbatim}
stockholm ! {add, lund, {lund, 'sweden@lingon'}},
  lund ! {add, stockholm, {stockholm, 'sweden@lingon'}},
  ...
\end{verbatim}

The network was initialized by letting all routers broadcast their link-state messages and subsequently update their routing
 tables.

\section{Evaluation}

\textit{In the graph described above a message have multiple possible routes between all cities.}
The evaluation consisted of sending messages between certain routers and tracking the messages ways through the network.

These are the messages sent:
\begin{enumerate}
  \item Stockholm -$>$ Göteborg
  \item Göteborg -$>$ Stockholm
  \item Örebro -$>$ Stockholm
  \item Örebro -$>$ Lund
\end{enumerate}

This is the log from the test procedure:

\begin{tiny}
\begin{verbatim}
Message 1:
stockholm: routing message (hejfransthlm) from stockholm to goteborg through lund
lund: routing message (hejfransthlm) from stockholm to goteborg through goteborg
goteborg: received message hejfransthlm

Message 2:
goteborg: routing message (hejfrangbg) from goteborg to stockholm through lund
lund: routing message (hejfrangbg) from goteborg to stockholm through stockholm
stockholm: received message hejfrangbg

Message 3:
orebro: routing message (hejfranorebro) from orebro to stockholm through uppsala
uppsala: routing message (hejfranorebro) from orebro to stockholm through stockholm
stockholm: received message hejfranorebro

Message 4:
orebro: routing message (hejfranorebro) from orebro to lund through goteborg
goteborg: routing message (hejfranorebro) from orebro to lund through lund
lund: received message hejfranorebro
\end{verbatim}
\end{tiny}

\section{Conclusions}

\textit{All messages took the correct route.}
According to the systems logging output, each message took the shortest route through the network to the correct host.
The experiment is very limited in both sample size and complexity. To create a more realistic example, the network
should consist of a larger number of routers, and more importantly investigate the impact of joining or leaving routers,
configuration changes and undirected edges in the graph.

\end{document}
